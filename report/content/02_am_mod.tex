\section{Modulación AM}

Para el circuito de la figura entregada:

\begin{figure}[ht!]
    \centering
    \begin{tikzpicture}
        % Paths, nodes and wires:
        \draw (1, 5.5) to[sinusoidal voltage source, l={$V_i$}, label distance=0.02cm] (1, 3.5);
        \draw (4, 3) to[sinusoidal voltage source, l={$V_c$}, label distance=0.02cm] (4, 1);
        \draw (2, 6) to[american resistor, l={$R$}, label distance=0.02cm] (5, 6);
        \draw node[op amp] at (6.19, 5.51) {};
        \draw (5, 5.02) to[american resistor, l={$R/2$}, label distance=0.02cm] (2, 5);
        \draw (1, 5.5) -| (2, 6);
        \draw (2, 5.5) -| (2, 5);
        \draw (5, 7) to[american resistor, l={$R$}, label distance=0.02cm] (7, 7);
        \draw (5, 7) -| (5, 6);
        \draw (7, 7) -| (7.38, 5.51);
        \draw node[nmos] at (5, 2.98) {};
        \draw (5, 5) -| (5, 3.75);
        \draw (5, 2.21) -| (5, 1);
        \draw node[sground] at (5, 1) {};
        \draw node[sground] at (4, 1) {};
        \draw node[sground] at (1, 3.5) {};
        \draw (7.38, 5.51) -| (8, 5.5);
        \draw node[ocirc] (N1) at (8, 5.5) {} node[anchor=west] at (N1.east){$V_o$};
        \draw node[circ] (N2) at (5, 5) {} node[anchor=south east] at (N2.north west){$V_+$};
        \draw node[circ] (N3) at (5, 6) {} node[anchor=south east] at (N3.north west){$V_-$};
    \end{tikzpicture}
    \caption{Circuito de Modulación AM}
    \label{fig:am_mod_ckt}
\end{figure}

Se puede hacer un análisis según el estado del transistor, abierto y cerrado.

\paragraph{Transistor abierto}

Como el transistor abierto significa que no fluye corriente por $R/2$, entonces $V_+ = V_i = V_-$ (asumiendo cortocircuikto virtual).

De lo anterior, como $V_-$ y $V_i$ tienen igual voltaje, no hay corriente pasando por $R$, lo que significa que $V_o$ va a ser igual a $V_i$.

\paragraph{Transistor cerrado}

Como el transistor cerrado significa un cortocircuito, se ve que $V_+ = 0$ (tierra) y de esta manera $V_- = 0$ y la corriente a través de $R$ es