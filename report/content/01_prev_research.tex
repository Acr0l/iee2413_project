\section{Investigación previa}

\subsection{La Transformada de Fourier Discreta}

La DFT de una señal discreta $x[n]$, de longitud $N$, está definida como:

\begin{equation}
    X[k] = \sum_{n=0}^{N-1} x[n] \cdot e^{-j \frac{2\pi}{N}kn}, \quad k = 0, 1, ..., N-1
\end{equation}

El cálculo directo de esta fórmula requiere $\mathcal{O}(N^2)$ operaciones complejas.

\paragraph{Idea general del algoritmo Cooley-Tukey:}

El algoritmo Cooley-Tukey se basa en el paradigma \textit{divide and conquer}, y divide la DFT de tamaño $N$ (donde $N$ es potencia de dos, es decir, $N=2^m$) en dos DFTs de tamaño $N/2$:

\begin{itemize}
    \item Una que contiene los elementos en posiciones pares: $x[0], x[2], x[4], \dots$
    \item Otra con los elementos en posiciones impares: $x[1], x[3], x[5], \dots$
\end{itemize}

Utilizando esta separación, se puede reescribir la DFT como:

\begin{equation}
    X[k] = \sum_{n=0}^{N/2 - 1} x[2n] \cdot W_N^{2kn} + \sum_{n=0}^{N/2 - 1} x[2n+1] \cdot W_N^{(2n+1)k}
\end{equation}

donde $W_N = e^{-j\frac{2\pi}{N}}$ es la raíz $N$-ésima de la unidad.

Agrupando términos:

\begin{equation}
    X[k] = E[k] + W_N^k \cdot O[k]
\end{equation}
\begin{equation}
    X[k + N/2] = E[k] - W_N^k \cdot O[k]
\end{equation}

donde $E[k]$ es la FFT de los elementos pares y $O[k]$ la FFT de los impares. Esta descomposición se aplica recursivamente hasta que se obtienen DFTs de tamaño 1.

\paragraph{Etapas del algoritmo Radix-2:}

\begin{enumerate}
    \item \textbf{Bit-reversal:} Reordenamiento de los datos de entrada según el orden inverso de los bits del índice binario.
    \item \textbf{Cálculo en etapas:} Se realizan $\log_2 N$ etapas, cada una combinando pares de subproblemas más pequeños usando operaciones llamadas \textit{butterflies}.
    \item \textbf{Butterfly operation:} Para cada par $(a, b)$ y una raíz $W_N^k$, se computa:
          \begin{equation}
              a' = a + W_N^k \cdot b, \quad b' = a - W_N^k \cdot b
          \end{equation}
\end{enumerate}

\paragraph{Complejidad computacional:}

Gracias a esta estructura recursiva, el algoritmo tiene una complejidad de:
\begin{equation}
    \mathcal{O}(N \log_2 N)
\end{equation}
lo que representa una mejora sustancial con respecto a la DFT directa.

\paragraph{Ejemplo para $N=4$:}

Sea $x = [x_0, x_1, x_2, x_3]$, se procede como sigue:

\begin{itemize}
    \item FFT de pares: $E[k] = x_0 + x_2 \cdot W_2^k$
    \item FFT de impares: $O[k] = x_1 + x_3 \cdot W_2^k$
\end{itemize}

\noindent Combinación:
\begin{equation}
    X[0] = E[0] + W_4^0 \cdot O[0], \quad X[1] = E[1] + W_4^1 \cdot O[1]
\end{equation}
\begin{equation}
    X[2] = E[0] - W_4^0 \cdot O[0], \quad X[3] = E[1] - W_4^1 \cdot O[1]
\end{equation}
